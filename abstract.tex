\begin{center}
    Abstract of the Dissertation  \\
    \vspace{0.2cm}
    \textbf{Transmitter Localization and Optimizing Initial State in Classical/Quantum Sensor Networks}  \\
    \vspace{0.2cm}
    by    \\
    \vspace{0.2cm}
    \textbf{Caitao Zhan}  \\
    \vspace{0.2cm}
    \textbf{Doctor of Philosophy}  \\
    \vspace{0.2cm}
    \textsc{in}  \\
    \vspace{0.2cm}
    \textbf{Computer Science}   \\
    \vspace{0.2cm}
    Stony Brook University    \\
    \vspace{0.2cm}
    \textbf{2024}
\end{center}

In shared spectrum systems, it is important to be able to localize simultaneously present multiple intruders 
(unauthorized transmitters) to effectively protect a shared spectrum from malware-based, jamming, 
or other multi-device unauthorized-usage attacks. We address the problem of localizing multiple intruders using 
a distributed set of classical radio-frequency (RF) sensors in the context of a shared spectrum system. 
In contrast to single transmitter localization, multiple transmitter localization (MTL) has not been thoroughly studied. 
The key challenge in solving the MTL problem comes from the need to “separate” an aggregated signal received from 
multiple intruders into separate signals from individual intruders. We solve the problem via a Bayesian-based approach 
and a deep-learning-based approach. 

After addressing multiple transmitter localization with a network of classical RF sensors, 
we explore using a quantum sensor network for transmitter localization. 
A quantum sensor network is a network of spatially dispersed sensors that leverage the quantum superposition and quantum entanglement. We pose our transmitter localization problem as 
a quantum state discrimination (QSD) problem and use the positive operator-valued measurement as a tool 
for localization in a novel way. Then, we address the additional challenge of the impracticality 
of general quantum measurement by developing new schemes that replace the QSD's measurement operators 
with trained parameterized hybrid quantum-classical circuits. 

Finally, we investigate problems that are unique in quantum sensors. We look into optimizing the initial state of detector sensors 
in quantum sensor networks. We consider a network of quantum sensors, where each sensor is a qubit detector that “fires”, 
i.e., its state changes when an event occurs close by. The change in state due to the firing of a detector is given 
by a unitary operator. The determination of the firing sensor can be posed as a QSD problem which incurs a probability 
of error depending on the initial state and the measurement operators used. We address the problem of determining 
the optimal initial state of the quantum sensor network that incurs a minimum probability of error in determining the firing sensor.
The optimal initial state is in general an entangled state, and thus there is a demand to generate and distribute the 
entangled state to a network of sensors. The last part of the thesis is about the efficient generation and distribution of entangled pairs.
