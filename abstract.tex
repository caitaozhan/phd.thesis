\begin{abstract}
\addchaptertocentry{\abstractname} % Add the abstract to the table of contents


In shared spectrum systems, it is important to be able to localize simultaneously present multiple intruders 
(unauthorized transmitters) to effectively protect a shared spectrum from malware-based, jamming, or other 
multi-device unauthorized-usage attacks.
We address the problem of localizing multiple intruders using a distributed set of classical radio-frequency (RF) sensors 
in the context of a shared spectrum system. 
In contrast to single transmitter localization, multiple transmitter localization (MTL) has not been thoroughly studied.  
The key challenge in solving the MTL problem comes from the need to “separate” an aggregated
signal received from multiple intruders into separate signals from individual intruders.
We solve the problem via a Baysian-based approach and a deep-learning-based approach.

After addressing multiple transmitter localization with a network of classical sensors, we explore a network of quantum sensors
and continue the work of transmitter localization using the quantum sensors.
Quantum sensor network is a network of spatially dispersed sensors that leverage the quantum properties of light and matter, e.g., quantum coherance and quantum entanglement.
We pose our transmitter localization problem as a quantum state discrimination problem and use the positive operator-valued measurement (POVM) as a tool for localization in a novel way.
Quantum entanglement is a critical resource for the task of distributed quantum sensing.
So we also investigate an efficient way to distribute entangled pairs (EPs).
Distributing EPs is challenging because of the no-cloning theorem and the long-distance direct transmission of qubit states being infeasible due to unrecoveralble errors.
We develop a heuristic algorithm that efficiently route EPs in a quantum network.

For the proposed work, we plan to keep the investigation of transmitter localization with a quantum sensor network.
POVM is the current quantum measurement we are using, it is general but not very practical. 
Instead of POVM, we plan to use the projective measurement in the computational basis.
To better optimize the measurement process, parameterized quantum circuit (quantum machine learning) will be utilized.


\end{abstract}