\chapter{Introduction}
\label{chap:intro}

\section{Background and Motivation}

\emph{Wireless sensor network} (WSN) is a network of spatially dispersed and dedicated sensors that monitor and record the 
physical conditions of the environment and forward the collected data to a central location.
WSN can measure environment conditions such as temperature, sound, pollution levels, humidity, wind and radio spectrum.
WSN refers to classical sensor networks since everything it involves is classical.
A sensor network becomes a \emph{quantum sensor network} (QSN) when the sensors leverage some quantum properties of light and matter, 
such as quantum coherance and quantum entanglement.
Quantum sensors are extremely sensitive to physical quantities such as magnetic field, electric field, quadrature displacement
and phase shift in the optic field.

\para{Classical.} WSNs have various applications.
In this thesis, the application we focus on is spectrum surveillance and monitoring for security and threat detection.
The \emph{core problem} involved in this application is \emph{transmitter localization}, and in particular, multiple transmitter localization (\mtl) as 
the number of transmitter present in an area could be more than one and localizing multiple transmitters are not independent.
The reason for being not independent is that a sensor receives an aggregated power from multiple transmitters and separating 
the power from different multiple sources is impractical.
That an aggregated received power not able to separete is a big challenge for \mtl.
Furthermore, in a shared spectrum paradigm, presence of an evolving set of authorized users 
(e.g., primary and secondary users) adds to the challenge.

The shared spectrum paradigm composes an important background for our \mtl work.
The RF spectrum is a natural resource in great demand due to the unabated increase in mobile (and hence, wireless) data consumption~\cite{Jeffrey14}. 
The research community has addressed this capacity crunch via development of shared spectrum paradigms, wherein the spectrum 
is made available to unlicensed users (secondaries) as long as they do not interfere with the transmission of licensed incumbents (primaries).
The fundamental objective behind such shared spectrum paradigms is to maximize spectrum utilization,
the viability of such systems depends on the ability to effectively guard the shared spectrum against unauthorized usage. 
The current mechanisms however to locate such unauthorized users (intruders) are human-intensive and time-consuming, 
involving FCC enforcement bureau which detects violations via complaints and manual investigation~\cite{mobicom17-splot}. 

Motivated by above, we seek for an effective technique that is able to accurately localize multiple simultaneous
intruders and even in the presence of dynamically changing set of authorized users.
Our solution assumes a network of crowdsourced sensors wherein relatively low-cost spectrum sensors are available
for gathering signal strength in the form of received power.
We introduce two different approaches to the \mtl problem.
The first approach is a hypothesis-driven Bayesian approach, viz. maximum a posterior approach, where wherein each hypothesis is a configuration
(i.e. a combination of $\langle$location, power$\rangle$ pair of the potential intruders), and the goal is to determine the hypothesis 
that best explains the sensor observations.
The second approach is a deep learning-based approach. First, we encode the sensors' observation data into an image.
Then, we frame \mtl as a sequence of two steps: image-to-image translation 
and object detection, each of which is solved using a trained CNN model. 
The first step of image-to-image translation maps an input image representing sensor readings to an image
representing the distribution of transmitter locations, and the second object detection step derives precise locations of
transmitters from the image of transmitter distributions. 
Besides the location, the transmission power is another property of a transmitter that we wish to estimate.
We introduces some novel methods to estimate the power of multiple transmitters.
We also introduce a novel interapolation method for received signal strength.


\para{Quantum.}
In the quantum side, we use QSN, instead of WSN, to continue solving the problem of transmitter localization.
Albeit classical sensors are omnipresent, there are big motivations to explore quantum sensors.
Quantum sensing is an emerging field that leverages the quantum properties of light and matter at atomic/subatomic scales and has the potential to sense signals at an unprecedented level of precision.
Quantum sensing brings new opportunities to new and well-established problems.
For example, physicists in the year 2016 used squeezed quantum states to improve the sensitivity of the Laser Interferometer Gravitational-wave Observatory (LIGO) detector and successfully detected gravitational waves.
In~\cite{PRL20-qsn}, researchers use some distributed quantum RF-photonic sensors to estimate the amplitude and phase of a radio signal.
They showed the performance of sensing a global property of the RF wave is enhanced by leveraging a shared multipartite entangled state produced by squeezed light.
In their experiments, the estimation variance of RF amplitude and phase both beat the standard quantum limit by over 3 dB.
The precision improvement factor of $1/\sqrt{N}$ for $N$ sensors is known as reaching the Heisenberg limit.

Motivated by the above, we aim to leverage quantum sensors to perform some canonical tasks and thus open a new avenue of research.
The canonical task we picked is RF transmitter localization~\cite{nsdi13-arraytrack,pmc22-deepmtlpro}.
We consider a network of quantum sensors distributed in a geographic area and a single transmitter active in the area to be localized.
We pose the localization problem as a quantum state discrimination problem~\cite{bergou-review-2007}. 
In our approach, the quantum sensor network reports a quantum state, and we discriminate the quantum state via 
positive-operator valued measure (POVM) and the POVM's output indicates the transmitter location.
The key challenge here is the scalability challenge, i.e., the method's time and space complexity grows exponentially against the
number of sensors and the method's localization accuracy decrease against the number of discrete locations.
To solve the challenge, we propose a two-level POVM method that is comprised of a coarse-level POVM and a fine-level POVM.
The two level idea is effective and can be generalized into three levels and more.

Quantum entanglement is a phenomenon that has no counterpart in the classical world.
It is the physical phenomenon that occurs when a group of particles (electrons, photons, etc) are generated, interact, or share spatial proximity in a way such that
the quantum state of each particle of the group cannot be described independently of the state of the others, including when the particles
are separated by a large distance.
In short, quantum entanglement is a specially strong correlation between multiple particles.
In our context of QSNs, entanglement can be served as a resource to enhance the perforance of the QSN.
For example, the $1 / \sqrt{N}$ improvement factor mentioned above requires the initial probe state as an entangled state.
The entanglement pair resource is generated at a single node, but the quantum sensors are spatially distributed.
Thus, a major problem is to distribute (or route) the entanglement to the quantum sensors at a potentially large distance apart.
This a challenging problem in the field of quantum communication.
Physical transmission of quantum states accross nodes can incur irreparable communication errors, as the no-cloning theorem proscribes
making independent copies of arbitrary qubits.
The establish of entanglement over long distance is challenging due to the low probability of success of the underlying physical process
(short distance entanglement and swapping).
In this thesis, we propose an efficient heuristic approach that efficiently route an entanglement pair in a quantum network.


\section{Thesis Statement}

The thesis statement is: \textbf{Transmitters can be localized efficiently and accurately with our designed methods in classical and quantum sensor networks.} 
Consider some transmitters to be localized a geographpical area. We deploy a distributed set of classical or quantum sensors in the area.
We aim to efficently and accurately localize the transmitters by processing the data received from the sensors intelligently.
Besides the core goal of transmitter localization, we also solve closely related problems such as transmission power estimation, 
receive signal strength interpolation and quantum network routing of entanglement.

\section{Contributions and Organization of this Thesis}

Towards the goal of our thesis we make the following contributions:

\begin{itemize}
    \item In Chapter~\ref{chap:ipsn}, we introduce an efficient hypothesis-based Bayesian approach \ouralgo for multiple transmitter localization (\mtl) problem (Section~\ref{sec:time}); 
          A closed-form equation for the estimation of transmission power (Section~\ref{sec:ipsn-power});
          A novel received signal strength interpolation method inspired from the power law distribution (Section~\ref{sec:inter});
          Extend \ouralgo to accommodate the presence of authorized users (Section~\ref{sec:auth}).
    \item In Capter~\ref{chap:wowmom-pmc}, we introduce a deep learning-based approach \our for the \mtl problem (Section~\ref{sec:translate}, Section~\ref{sec:detect});
          Extend \our via deep learning models to accommodate the presense of authorized users (Section~\ref{sec:authorized});
          A deep learning-based approach that estimates the transmission power of multiple transmitters (Section~\ref{sec:power}).
    \item In Chapter~\ref{chap:icc}, we introduce the concept of quantum sensor networks and the model of a quantum sensor (Section~\ref{sec:quantum_problem});
          In the context of quantum sensor networks, we pose a transmitter localization problem as a quanum state discrimination problem 
          and introduce a novel quantum localization method \povm and \povmpro based on positive-operator valued measure (Section~\ref{sec:povm}).
    \item In Chapter~\ref{chap:swappingtrees}, we introduce an efficient heuristic algorithm \dpalt for routing an entanglement pair.
          The algorithm is Dijkstra-based, and the path selection metric is a closed-form expression that models a path as a tree near accurately (Section~\ref{chap:swappingtrees}).
\end{itemize}


\section{Proposed: Transmitter Localization in QSN with Measurement in the Computational Basis}
In Chapter~\ref{chap:icc}, we pose the transmitter localization problem as a quantum state discrimination problem
and measure the quantum state reported by the quantum sensor network (QSN) via a positive-operator valued measure (POVM).
Although POVM is good in theory, it is hard to implement in practice.
Current hardware such as the IBM quantum computer only support measurement in the computational basis.

For the proposed work, we continue the work of transmitter localization via quantum state discrimination in a more practical way.
Instead of measuring via a POVM (square root measurement), we aim to \textit{measure in the computational basis}.
To optimize the measurement, we apply \textit{parameterized quantum circuits} (quantum machine learning) before the computational measurement gates.
We would like to run the measurement not only in simulations, but on \textit{real IBM quantum computers}.
We are currently in the noisy intermediate-scale quantum (NISQ). 
Thus, we need to consider \textit{noise} in the measurement process, which is ignored in work of Chapter~\ref{chap:icc}.

