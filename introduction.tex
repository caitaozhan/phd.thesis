\chapter{Introduction}
\label{chap:intro}

\section{Background}

\emph{Wireless sensor network} (WSN) is a network of spatially dispersed and dedicated sensors that monitor and record the 
physical conditions of the environment and forward the collected data to a central location.
WSN can measure environment conditions such as temperature, sound, pollution levels, humidity, wind and radio spectrum.
WSN refers to classical sensor networks since everything it involves is classical.
A sensor network becomes a \emph{quantum sensor network} (QSN) when the sensors leverage some quantum properties of light and matter, 
such as quantum coherance and quantum entanglement.
Quantum sensors are extremely sensitive to physical quantities such as magnetic field, electric field, quadrature displacement
and phase shift in the optic field.

\para{Classical.} WSNs have various applications.
In this thesis, the application we focus on is spectrum surveillance and monitoring for security and threat detection.
The \emph{core problem} involved in this application is \emph{transmitter localization}, and in particular, multiple transmitter localization (\mtl) as 
the number of transmitter present in an area could be more than one and localizing multiple transmitters are not independent.
The reason for being not independent is that a sensor receives an aggregated power from multiple transmitters and separating 
the power from different multiple sources is impractical.
That an aggregated received power not able to separete is a big challenge for \mtl.
Furthermore, in a shared spectrum paradigm, presence of an evolving set of authorized users 
(e.g., primary and secondary users) adds to the challenge.

The shared spectrum paradigm composes an important background for our \mtl work.
The RF spectrum is a natural resource in great demand due to the unabated increase in mobile (and hence, wireless) data consumption~\cite{Jeffrey14}. 
The research community has addressed this capacity crunch via development of shared spectrum paradigms, wherein the spectrum 
is made available to unlicensed users (secondaries) as long as they do not interfere with the transmission of licensed incumbents (primaries).
The fundamental objective behind such shared spectrum paradigms is to maximize spectrum utilization,
the viability of such systems depends on the ability to effectively guard the shared spectrum against unauthorized usage. 
The current mechanisms however to locate such unauthorized users (intruders) are human-intensive and time-consuming, 
involving FCC enforcement bureau which detects violations via complaints and manual investigation~\cite{mobicom17-splot}. 

Motivated by above, we seek for an effective technique that is able to accurately localize multiple simultaneous
intruders and even in the presence of dynamically changing set of authorized users.
Our solution assumes a network of crowdsourced sensors wherein relatively low-cost spectrum sensors are available
for gathering signal strength in the form of received power.
We introduce two different approaches to the \mtl problem.
The first approach is a hypothesis-driven Bayesian approach, viz. maximum a posterior approach, where wherein each hypothesis is a configuration
(i.e. a combination of $\langle$location, power$\rangle$ pair of the potential intruders), and the goal is to determine the hypothesis 
that best explains the sensor observations.
The second approach is a deep learning-based approach. We frame \mtl as a sequence of two steps: image-to-image translation 
and object detection, each of which is solved using a trained CNN model. 
The first step of image-to-image translation maps an input image representing sensor readings to an image
representing the distribution of transmitter locations, and the second object detection step derives precise locations of
transmitters from the image of transmitter distributions. 
Besides the location, the transmission power is another property of a transmitter that we wish to estimate.
We also introduces some novel methods to estimate the power of multiple transmitters.

\para{Quantum.}
In the quantum side, we use QSN, instead of WSN, to continue solving the problem of transmitter localization.
Albeit classical sensors are omnipresent, there are big motivations to explore quantum sensors.
Quantum sensing is an emerging field that leverages the quantum properties of light and matter at atomic/subatomic scales and has the potential to sense signals at an unprecedented level of precision.
Quantum sensing brings new opportunities to new and well-established problems.
For example, physicists in the year 2016 used squeezed quantum states to improve the sensitivity of the Laser Interferometer Gravitational-wave Observatory (LIGO) detector and successfully detected gravitational waves.
In~\cite{PRL20-qsn}, researchers use some distributed quantum RF-photonic sensors to estimate the amplitude and phase of a radio signal.
They showed the performance of sensing a global property of the RF wave is enhanced by leveraging a shared multipartite entangled state produced by squeezed light.
In their experiments, the estimation variance of RF amplitude and phase both beat the standard quantum limit by over 3 dB.
The precision improvement factor of $1/\sqrt{N}$ for $N$ sensors is known as reaching the Heisenberg limit.
Motivated by the above, we aim to leverage quantum sensors to perform some canonical tasks and thus open a new avenue of research.
The canonical task we picked is RF transmitter localization~\cite{nsdi13-arraytrack,pmc22-deepmtlpro}.
We consider a network of quantum sensors distributed in a geographic area and a single transmitter active in the area to be localized.
We pose the localization problem as a quantum state discrimination problem~\cite{bergou-review-2007}. 
In our approach, the quantum sensor network reports a quantum state, and we discriminate the quantum state via 
positive-operator valued measure (POVM) and the POVM's output indicates the transmitter location.

Quantum entanglement is a phenomenon that has no counterpart in the classical world.
It is the physical phenomenon that occurs when a group of particles (electrons, photons, etc) are generated, interact, or share spatial proximity in a way such that
the quantum state of each particle of the group cannot be described independently of the state of the others, including when the particles
are separated by a large distance.
In short, quantum entanglement is a specially strong correlation between multiple particles.
In our context of QSNs, entanglement can be served as a resource to enhance the perforance of the QSN.
For example, the $1 / \sqrt{N}$ improvement factor mentioned above requires the initial probe state as an entangled state.
The entanglement pair resource is generated at a single node, but the quantum sensors are spatially distributed.
Thus, a major problem is to distribute (or route) the entanglement to the quantum sensors at a potentially large distance apart.
This a challenging problem in the field of quantum communication.
Physical transmission of quantum states accross nodes can incur irreparable communication errors, as the no-cloning theorem proscribes
making independent copies of arbitrary qubits.
The establish of entanglement over long distance is challenging due to the low probability of success of the underlying physical process
(short distance entanglement and swapping).
In this thesis, we propose an efficient heuristic approach that efficiently route an entanglement pair in a quantum network.


\section{Thesis Statement}

The thesis statement is: \textbf{Transmitters can be localized efficiently and accurately with our designed methods in classical and quantum sensor networks.} 
Consider some transmitters to be localized a geographpical area. We deploy a distributed set of classical or quantum sensors in the area.
We aim to efficently and accurately localize the transmitters by processing the data received from the sensors intelligently.
Besides the core goal of transmitter localization, we also solve closely related problems such as transmission power estimation
and quantum network routing of entanglement.

\section{Organization and Contributions of this Thesis}

Towards the goal of our thesis we make the following contributions:

\begin{itemize}
    \item Improve the scarce spectrum usages of a wireless network
    \item Introduce the optimal communications in quantum networks
    \item Improving the high latency of quantum communications
    \item Devising effective ways of distributing multipartite entanglement in a quantum network (Proposed)
\end{itemize}

\vspace{6pt}\noindent \textbf{Improve the scarce spectrum usages of a wireless network (SenSys'22, under review)}

Shared spectrum systems facilitate spectrum allocation to unlicensed users without harming the licensed users; they offer great promise in optimizing spectrum utilization, but their management is challenging.
Current allocation schemes are either based on imperfect propagation models or poor spatial granularity spectrum sensors which leads to spectrum under-utilization,  the fundamental objective of shared
spectrum systems.
% To allocate spectrum near-optimally to secondary users (SUs) in general scenarios, we 
% fundamentally need to have knowledge of the signal path-loss function. In practice, however, even the best known path-loss models have unsatisfactory accuracy, and conducting extensive surveys to gather path-loss values is infeasible.
%%%%%%%%%%%%%%%

In this work, we propose to learn the spectrum allocation function directly using supervised learning techniques. 
For settings where information about the primary users' may not be available, we propose to use a crowdsourced sensing architecture and use the spectrum sensor readings as features. 
%%%%%%%%%%%%%
For spectrum allocation to a single SU, we develop a CNN-based approach
and address various challenges that arise in our context; to handle multiple SU
requests simultaneously, we extend our approach based on recurrent neural networks (RNNs).
%%%%%%
We present, via simulations and real testbed, the effectiveness of our developed techniques. We will show how easily our system outperforms the prior work by a high margin.


\vspace{6pt}\noindent \textbf{Introduce the optimal communications in quantum networks (TQE'22)}

Quantum network communication is challenging, as the No-cloning theorem in quantum regime makes many classical techniques inapplicable; in particular, 
direct transmission of qubit, quantum counterpart of classical bit, states over long distances is infeasible
due to unrecoverable errors.
%%%%%%%%%%%%%%%%%
For long-distance communication, the only viable communication approach is teleportation of quantum states, which requires a prior distribution of entangled pairs (\epss) of qubits.
Establishment of \epss across remote nodes can incur significant latency due to the low probability of success of the underlying physical processes. 

In this work, we develop efficient techniques that minimize \eps generation latency. While prior works
have focused on selecting entanglement \textit{paths}, we select \emph{entanglement swapping trees} which represent the entanglement generation structure more accurately.
Specifically, we develop a dynamic programming algorithm to select an optimal swapping-tree for a single pair of nodes, under the given capacity and fidelity constraints.
We also develop an efficient iterative algorithm to extract the best set of swapping trees for general setting; when the network tries to generate \epss across multiple set of pairs.
%  For the general setting,
% We present simulation results which show that our solutions outperform 
% the prior approaches by an order of magnitude and are viable for long-distance entanglement
% generation.



\vspace{6pt}\noindent \textbf{Improving the high latency of quantum communications (QCE'22)}
In this work, we propose and investigate a complementary technique to further reduce \eps generation latency---to pre-distribute \epss over certain (pre-determined) pairs of network nodes. 
Upon certain requests to the network, these re-distributed \epss can then be used to lower the generation latency.
% As mentioned before, \ep generation between a pair of node can be parallelized for lower latency. 
% This gives us an intuition to prepare some \eps across some nodes repeatedly such that many traffic requests can be served with.
%%%%

For such a pre-distribution approach to be most effective, we need to carefully select a set of node-pairs where
the \epss should be pre-distributed to minimize the generation latency of expected \eps requests, under a given cost constraint.
In this work,
we appropriately formulate the above optimization problem and design two
efficient algorithms, one of which is a greedy approach based on an 
approximation algorithm for a special case. 
The other approach is derived from the clustering methods.
%%%%%%%%
%%%%%%%%%%
Via extensive evaluations, we demonstrate the effectiveness of our approach and developed techniques;  
% we show that our developed algorithms outperform a naive approach by up to an order
% of magnitude. 


\vspace{6pt}\noindent \textbf{Devising effective ways of distributing multipartite entanglement in a quantum network (Proposed)} 
We also propose looking into the entanglement generation among more than two nodes, something that there is no classical counterpart. 
multipartite entanglement, physically, means there is a high correlation among all the participating nodes which has many applications from distributed computing to conference key sharing.
We will extend our previous works, which the focus was establishing bipartite entanglement between a pair of nodes.
As many other systems, entanglement generation is not noise free. To improve the quality service of quantum networks, we will take into consideration optimal distillation protocols merged into our multipartite entanglement.
Eventually, we try to propose a stack-based quantum network architecture where the different elements of the network is going to be orchestrated in an organic way.

