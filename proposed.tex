\chapter{Proposed: Transmitter Localization in QSN with Measurement in the Computational Basis}
\label{chap:proposed}

In Chapter~\ref{chap:icc}, we pose the transmitter localization problem as a quantum state discrimination problem
and measure the quantum state reported by the quantum sensor network (QSN) via a positive-operator valued measure (POVM).
Although POVM is good in theory, it is hard to implement in practice.
Current hardware such as the IBM quantum computer only support measurement in the computational basis.

For the proposed work, we continue the work of transmitter localization via quantum state discrimination in a more practical way.
In Chapter~\ref{chap:icc}, our measurement is done via a POVM constructed by equations for square root measurement~\cite{prettygood},
which is also called pretty good measurement. The measurement problem in the context of quantum state discrimination is an optimization problem
that can be formulated as a semidefinite programming problem (SDP)~\cite{semidefinite}. 
The pretty good measurement can be viewed as a relatively good heuristic for a measurement optimization problem.
However, it has two problems:
\begin{enumerate}
    \item Practical problem. POVM, the general measurement, is good in theory but implementing it in practice using a combination of
          single qubit rotation gate, CNOT gate and standard measurement gate (Fig.~\ref{fig:measurement}) is non-trivial. 
          In the literature, there are works~\cite{pra19-povm,PhysRevResearch.povm} that try to implement a single-qubit two/three element POVM and cliam
          that their solutions can be generalized to arbitrary number of qubits and arbitrary number of elements. 
          The solution (quantum circuits) in their work has a high circuit depth and requires ancillary qubits.
          Note that in Chapter~\ref{chap:icc}, a POVM of 8 qubits and up to 256 elements is needed, and using the 
          techniques from their work to implement the POVM we need looks daunting (at least to me).
          
    \item Performance problem. As its name ``pretty good'' suggests, it is ``okayish'', but not ``very good''.
          SDP solvers~\cite{diamond2016cvxpy} can optimally solve the measurement optimization problem,
          but the solver is not scalable as the number of qubits increases. 
          Note that a variable in the SDP formulation is a matrix (operator) whose dimension is the square of 2 to the power of number of qubits.
          Using sub-optimal pretty good measurement implies that there are always room for improvement.
\end{enumerate}

\begin{figure}[ht]
    \centering
    \includegraphics[width=0.15\textwidth]{figures/measurement-gate.png}
    \caption{The only measurement gate provided by the IBM Quantum Computer is a measurement in the standard basis, 
             also known as the z basis or computational basis. It can be used to implement any kind of measurement
             when combined with other gates.}
    \label{fig:measurement}
\end{figure}

To solve first problem of practicality, we plan to use the measurement under the standard computational basis, 
as in the quantum sensing protocol described in~\cite{RevModPhys.quantumsensing}.
Measurement in the computation basis is the only measurement operator provided by IBM quantum computers. See Fig.~\ref{fig:measurement}.

However, directly using the computational measurement will likely be far from the optimal measurement, even a lot worse than the pretty good measurement.
So this leads to serious a performance problem.
To solve it, we apply some quantum gates before the standard measurement gates to transform the qubits for better serving the computational measurement.
However, the design of the quantum gates is non-trivial. Take the example of single qubit rotational gate, it is hard to determine how large an angle to rotate.
Thus, we plan to resort to parameterized quantum circuits (also named quantum neural networks), wherein the angles of the rotation can be \textit{learned} through the training process.
Also note that we are in the noisy intermediate-scale quantum (NISQ) era. So we plan to take noise in to consideration, which is completely ignored in Chapter~\ref{chap:icc}.

Our end-goal is to be able to run the evaluation experiments on a \textit{real IBM quantum computer} using computational basis measurement and parameterized quantum circuits.
This is a huge leap compared with classical computer simulations done in Chapter~\ref{chap:icc}. 
We may encounter some unknown road blocks, but the journey is going to be fun.
