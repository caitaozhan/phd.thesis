\chapter{Acknowledgments}


I would like to express my sincere gratitude to my advisor, Prof. Himanshu Gupta, for his guidance and support throughout this journey. 
After chatting to many PhD students in the department about their advisors, I think Prof. Himanshu is one of the professors that provide 
the most support to and spend the most time on students, especially for tenured senior professors.
He has taught me how to conduct research, think scientifically, select good problems to solve, solve problems, write papers, and present the work.
When I run into obstacles, Prof. Himanshu can always help me find a way to overcome the obstacle.
This thesis came out to be ``half classical and half quantum'', due to the change of research direction from wireless networks to
quantum networks in the middle of my PhD. Quantum is hard and daunting, but I always have faith because I believe that 
Prof. Himanshu's ingenuity and vision will lead me to success. Thanks you, Himanshu.

I am also very thankful to my committee members, Professor Samir Das, CR Ramakrishnan, Mark Hillery and Nengkun Yu. 
Prof. Samir has big influence on me during the first half of PhD while I was working on spectrum sensing.
My PhD seniors told me to include Prof. Samir in the committee no matter what. 
Thereafter, he has been the chair of my RPE, Prelim, the Disertation committee.
Prof. CR has give me very insightful suggestions during the second half of my PhD while I was working on quantum networks.
His vast knowledge and intelligence helped me approach problems from different angles.
I am very fortunate to have the opportunity to collaborate with Prof. Mark, a very respected and famous scientist 
in the field of quantum optics and quantum physics. My work in quantum sensor networks is rooted from his expertise in quantum state discrimination.
It came out that Prof. Nengkun is also an expert in quantum state discrimination. I started ask him questions about quantum state discrimination right after
he moved from Australia to Stony Brook. Thus I am also grateful to Prof. Nengkun.

After thanking my advisor and committee members, I give thanks to my PhD peers. First and foremost, I would like to thank Mohammad Ghaderibaneh, a.k.a. Shahrokh.
We joined the lab at the same time in summer 2018 and closely collaborated on several projects since then. 
He is a very reliable teammate and I enjoyed the time we worked together.
When I just joined the lab in 2018, I collaborated with my lab senior Arani Bhattacharya on a paper and he set a very good example for me. 
Lab senior Jian Xu, Mallesh Dasari, and Santiago Vargas also gave me tremendous help. Other lab seniors including Yi, Qingqing, Max, and Shaifur are also important.
During the COVID year of 2020, Pranjal Sahu, Mallesh, and me are among the handful of students who still dare to go the department everyday in person, 
especially from July 2020 to early 2021 when there was no vaccine.
During that special time, Pranjal help me a lot on a project that requires training deep neural networks.
During the last months of my PhD, I collaborated with lab new member Xiaojie Fan extensively on a paper.
He is very hard working and I hope I have set a good example for him just like Arani had set a good example for me.
Other PhD peers worth mentioning include Tanmay, Prerna, Anand, Abeer, Weihai, Zhengyu, Duin, Salman, Pramodh, Ranjani, John, Bo, Manavjeet, Xiaoling, Yuhao, Yifei, Junyi, Yilun, Haotian, Yicheng, Yucheng, Tao, Zhepeng, Yu, Ting, and Fumi.
Outside of work, I would like to thank Peineng, Yilai, Zhenghong, Hongyu, Haoyan, Jason and Zeyu for time spend together on various hobbies, including kayaking, hiking, mountain biking, golf and soccer, making the PhD experience somewhat work-life balence.

In the end, I must give the most thanks to my parents, Yuhong Qian and Jianqiao Zhan, and also my uncle Yufei Qian 
and other family members including Beverley and Allen Fein.
In the beginning, my motivation of appling for a PhD program in computer science in the USA comes from my parents.
My father is a computer scientist and engineer himself, and he likes to share his ideas and viewpoints on various topics in computer science with me.
Without the support, knowledge and love from Yuhong and Jianqiao, I could never accomplish my PhD.
In the 1990s, my uncle Yufei was once an international student from China to USA entering a computer science graduate program.
20 years later in 2017, I followed the same path as my uncle. 
Yufei gave me enormous guidance on various topics in life. His life experience and wisdom is invaluable.
Allen and Beverley are my uncle and aunt in law who live in Long Island, NY and I have spend some wonderful times in their beautiful house in the Hamptons.
