\pagebreak

\section*{\centerline{{\textbf{Acknowledgments}}}}

\addcontentsline{toc}{chapter}{Acknowledgments}


I would like to express my sincere gratitude to my advisor, Prof. Himanshu Gupta, for his guidance and support throughout this journey. 
After chatting with many Ph.D. students in the department about their advisors, I think Prof. Himanshu is one of the professors who provide 
the most support to and spend the most time on students, especially for tenured senior professors.
He has taught me how to conduct research, think scientifically, select good problems to solve, solve problems, write papers, and present the work.
When I run into obstacles, Prof. Himanshu can always help me find a way to overcome the obstacle.
This thesis came out to be ``half classical and half quantum'', due to the change of research direction from wireless networks to
quantum networks in the middle of my Ph.D. Quantum is very hard, but I always have faith because I believe that 
Prof. Himanshu's ingenuity and vision will lead me to success. Thank you, Himanshu.

I am also very thankful to my committee members, Professor Samir Das, C.R. Ramakrishnan, Mark Hillery, and Nengkun Yu. 
Prof. Samir had a big influence on me during the first half of my Ph.D. while I was working on spectrum sensing.
My lab seniors told me to include Prof. Samir in the committee no matter what. 
Thereafter, he became the chair of my RPE, Prelim, and the dissertation committee.
Prof. C.R. gave me very insightful suggestions during the second half of my Ph.D. while I was working on quantum networks.
His vast knowledge and intelligence helped me approach problems from different angles.
I am very fortunate to have the opportunity to collaborate with Prof. Mark, a very respected and famous scientist 
in the field of quantum optics and quantum physics. My work in quantum sensor networks is rooted in his expertise in quantum state discrimination.
It came out that Prof. Nengkun is also an expert in quantum state discrimination. I started asking him questions about quantum state discrimination right after
he moved from Australia to Stony Brook. Thus I am also grateful to Prof. Nengkun.

After thanking my advisor and committee members, I give thanks to my Ph.D. peers. First and foremost, I would like to thank Dr. Mohammad Ghaderibaneh, a.k.a. Shahrokh.
We joined the lab at the same time in the summer of 2018 and closely collaborated on several projects since then. 
He is a very reliable teammate and I enjoyed the time we worked together.
When I just joined the lab in 2018, I collaborated with my lab senior Dr. Arani Bhattacharya on a paper and he set a very good example for me. 
Lab seniors Dr. Jian Xu, Dr. Mallesh Dasari, and Dr. Santiago Vargas also gave me tremendous help. Other lab seniors including Dr. Yi, Dr. Qingqing, and Dr. Max are also important.
During the COVID year of 2020, Dr. Pranjal Sahu, Mallesh, and I are among the handful of students who still dare to go to the department every day in person, 
especially from July 2020 to early 2021 when there was no vaccine.
During that special time, Pranjal helped me a lot on a project that required training deep neural networks.
During the last months of my Ph.D., I collaborated with lab new member Xiaojie Fan extensively on a paper.
He is very hard working and I hope I have set a good example for him just like Arani had set a good example for me.
Other Ph.D. peers worth mentioning include Tanmay, Prerna, Dr. Anand, Dr. Abeer, Weihai, Zhengyu, Dr. Duin, Pramodh, Ranjani, John, Bo, Manavjeet, Dr. Xiaoling, Yuhao, Yifei, Junyi, Yilun, Dr. Haotian, Yicheng, Yucheng, Tao, Zhepeng, Dr. Yu, Ting, and Fumi.
Outside of work, I would like to thank Peineng, Yilai, Zhenghong, Dr. Hongyu, Haoyan, Jason, and Zeyu for time spent together on various hobbies, including kayaking, hiking, mountain biking, soccer, and golf, making the Ph.D. experience somewhat work-life balance.

In the end, I must give the most thanks to my parents, Yuhong Qian and Jianqiao Zhan, and also my uncle Yufei Qian 
and other family members including Beverley and Dr. Allen Fein.
In the beginning, my motivation for applying for a Ph.D. program in computer science in the USA comes from my parents.
My father is a computer scientist and engineer himself, and he likes to share his ideas and viewpoints on various topics in computer science with me.
Without the support, knowledge, and love of Yuhong and Jianqiao, I could never accomplish my Ph.D.
In the 1990s, my uncle Yufei was once a Chinese international student at UNC enrolled in a computer science graduate program.
20 years later in 2017, I followed the same path as my uncle. 
Yufei gave me enormous guidance on various topics in life. His life experience and wisdom are invaluable.
Allen and Beverley are my uncle and aunt-in-law who live in Long Island, NY. I have spent some wonderful times in their beautiful house in the Hamptons.

