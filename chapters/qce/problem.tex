\section{Sensor Model, Problem, Related Work}
\label{sec:quantum_problem}

In this section, we start with motivating our choice of RF transmitter localization as an application for QSN, and then model the impact of an RF received signal on the quantum state of a quantum sensor. We then formulate the quantum localization problem and discuss related work.

\para{Motivation for Transmitter Localization.}
Accurate detection and localization of a wireless transmitter 
(typically, using a radio-frequency (RF) wireless signal) is important in a 
variety of wireless and/or mobile
applications, e.g.,
as an intruder detection in shared spectrum systems~\cite{arani2018}, localization of devices/users in
indoor settings (e.g., supermarkets, museums, virtual/augmented reality applications~\cite{sigcomm22-cyclops}), 
etc. In general,
transmitter localization is a key technology for location-based services and an 
improvement in transmitter localization will be very
beneficial to a variety of applications.
%%%%%%%%%%%%
In particular, 
in shared spectrum systems~\cite{arani2018}, there is a need to guard the shared spectrum
against unauthorized usage which entails detecting and localizing unauthorized transmitters that may
use and/or jam the spectrum illegally.
%%%%%%%%%%%%%%%%%%%%%
Classical techniques for transmitter localization involve triangulation~\cite{nsdi13-arraytrack} or fingerprinting techniques~\cite{infocom00-radar} (see~\cite{localization-survey} for a survey).
%%%%%%%%%%%%%%%%%%%%%%%%%%%%%%%%%%

Advances in quantum technologies have led to the creation of efficient quantum sensors for
radio-frequency (RF) signal detection that are much more sensitive than the classical 
antenna-based RF sensors and are expected to cover the entire 
RF spectrum~\cite{PhysRevApplied.rydberg}. 
E.g., in~\cite{PRL20-qsn},
researchers use some distributed entangled RF-photonic quantum
sensors to estimate the amplitude and phase of a radio
signal, and the estimation variance beats the standard quantum
limit by over 3 dB. 
%%%%%%%%%%%%%%%%
Thus, QSNs may have a great potential in accurate localization of wireless 
transmitters, which is a problem of great significance in many applications.

\para{Quantum Sensor Model}. 
Impact on a quantum sensor due to a physical phenomenon is typically modeled by an appropriate unitary operator that results in a quantum phase change~\cite{RevModPhys.quantumsensing}. 
Below, we model the change in the quantum phase
of a sensor's state due to an RF-received signal. 
Since the RF received signal (and thus the change in quantum phase) depends on the sensor's distance 
from the transmitter, we can use the phase change that occurs 
during the sensing period to localize an RF transmitter.

\softpara{Sensor's Hamiltonian.}
A quantum sensor's Hamiltonian $\hat{H}(t)$ is a sum of two\footnote{The third component 
of control Hamiltonian is chosen to tune the sensor in a controlled way~\cite{RevModPhys.quantumsensing}; we assume $\hat{H}_{control}=0$ in our analysis~\cite{egerstrom}.}
components~\cite{RevModPhys.quantumsensing}:
%%%%%%%%%%%%%%%%%%%%%%%%%%%%%%%%%%%%%%%%
$$\hat{H}(t) = \hat{H}_0 + \hat{H}_V(t)$$
where $\hat{H}_0$ is the internal Hamiltonian of the system and
$\hat{H}_V(t)$ is the change in the Hamiltonian due to an external signal $V(t)$.
%%%%%%%%%%%%%%%%%%%%%%%%%%
The internal Hamiltonian $\hat{H}_0$ remains fixed and is equal to $E_0 \ket{0}\bra{0} + E_1\ket{1}\bra{1}$, where $E_0$ and $E_1$ are energies corresponding to the 
$\ket{0}$ and $\ket{1}$ states respectively. 
%%%%%%%%%%%%%%%%%%%%%%
The signal Hamiltonian $\hat{H}_V(t)$ is given by:\footnote{Here, we ignore the 
transverse component of $\hat{H}_V(t)$~\cite{egerstrom}, since, in most sensor applications, the
energy difference $\Delta E = E_1 - E_0$ is much higher than the energy 
changes introduced by the signal $V(t)$~\cite{RevModPhys.quantumsensing}.}
%%%%%%%%%%%%%%%%%%%%%%%%%%%%%%%%%%%
$$\hat{H}_{V}(t) =  -\frac{1}{2}\gamma V_{\parallel}(t)\hat{\sigma}_z$$
where $\sigma_z$ is the Pauli-Z matrix, $V_{\parallel}(t)$ is the parallel component of the signal $V(t)$, and $\gamma$ is the coupling of the qubit to the parallel component.
%%%%%%%%%%%%%%%%%%%%%%%%%%%%%%%%%%%%%
In essence, the above induces a change in the spin in the $z$ axis direction resulting
in a  qubit \emph{phase shift}. Above, 
${V_{\parallel}}(t)$ at the sensor is given by:
$$V_{\parallel}(t) = E \sin(2\pi f  t + \theta)$$
where $E$ is the signal's (electric field) maximum amplitude, $f$ is the signal 
frequency and $\theta$ is the signal's phase.

\softpara{Evolution Unitary Operator.} Assume at time $t=0$, the quantum state is $\ket{\phi_0}$. Then at time $t=t'$, the state $\ket{\phi_{t'}}$ is,
$$ \ket{\psi_{t'}} = \hat{U}(0, t') \ket{\psi_0} $$
where the time evolution unitary operator $\hat{U}(0, t')$ due to the signal is given by:
\begin{eqnarray*}
\hat{U}(0, t') &=& e^{\frac{i}{\hbar} \int_{0}^{t'} \hat{H}_V(t) dt}  \\
&=& e^{\frac{i}{\hbar} \int_{0}^{t'} (-\frac{1}{2}\gamma V_{\parallel}(t)\hat{\sigma}_z) dt} 
% \\
% &=& e^{\frac{i}{\hbar} \int_{0}^{t'} (-\frac{1}{2}\gamma V_{\parallel}(t)\hat{\sigma}_z) dt}
\end{eqnarray*}
where $\hbar=6.626\times 10^{-34} J\cdot s$ is the plank constant.
The unit of coupling $\gamma$ is $J/(V\cdot m^{-1})$, and the unit of $V_{\parallel}(t)$ is $V\cdot m^{-1}$.
% Recall that $\hat{H}_0$ is fixed.
%\red{and can be viewed as generating a global phase, thus we ignore it during our computation}.

\softpara{Phase Shift over a Sensing Time Window.}
Let us represent $\hat{U}(0, t')$ as~\cite{nature21_phase,Zhang_2021}
\begin{equation}
    \hat{U}(0, t') = e^{-\frac{i}{2} \phi  \hat{\sigma}_z} \label{eqn:unitary}
\end{equation}
where the phase shift $\phi  =  \int_{0}^{t'} \frac{\gamma}{\hbar} V_{\parallel}(t) dt$, 
accumulated during the sensing time $[0, t']$ due to the signal $V(t)$ is estimated as follows.
%%%%%%%%%%%%%%%%%%%%%%%%%%%%%%
Note that $V_{\parallel}(t)$ is a sinusoidal function---and hence, the phase shift in
one full cycle ($t' =\sfrac{1}{f}$) is zero. To address this, we invert the qubit whenever the sinusoidal function turns from positive to negative using a $\pi$ pulse~\cite{RevModPhys.quantumsensing}.
%%%%%%
Thus, the accumulated phase in one cycle $\phi = \int_{0}^{\sfrac{1}{f}} \frac{\gamma}{\hbar} V_{\parallel}(t) dt = \frac{2}{\pi \hbar} \gamma E \frac{1}{f}$.
Since the sensing time $t'$ is expected to be much larger than $1/f$, the phase
shift accumulated during the sensing time $[0, t']$ can be estimated by:
\begin{equation}
    \phi =  \frac{2}{\pi \hbar} \gamma E t'
    \label{eqn:phase_shift}
\end{equation}
Thus, for a fixed sensing time duration, the phase shift in the sensor's quantum state
accumulated due to the signal is proportional to $E$, the signal's maximum amplitude, 
which is a function of the distance from the transmitter (see~\S\ref{sec:quantum_eval}). 
%%%%%%

\softpara{Impact on Multiple Quantum Sensors.}
Consider a set of $m$ quantum sensors distributed over an area, with a global $m$-qubit quantum state of $\ket{\psi_0}$. 
Consider a transmitter at a certain specific location in the area. 
Let $\hat{U}_i$ be
the impact on the $i^{th}$ sensor due to the transmitter over a sensing time window. 
Then, the overall impact on the {\em global} quantum state is represented by a \emph{tensor product}  of $m$ individual unitary operators, i.e., $\bigotimes_{i=1}^{m} \hat{U}_i$, and
the evolved global state state is $\bigotimes_{i=1}^{m} \hat{U}_i\ket{\psi_0}$. 

\para{Problem Definition.}
Consider a network of quantum sensors distributed in a geographic area 
and a potential transmitter/intruder in the area.
%%%%%%%%%%%
Let the initial state of the system of quantum sensors network 
be $\ket{\psi_{0}}$. 
%%%%%%%%%%
As described above, due to the transmission from the intruder, the quantum state evolves to $\ket{\psi_{t'}} = \hat{U} \ket{\psi_{0}}$ over a period of time $t'$.
The transmitter localization problem is to determine the location of the transmitter based on
the evolved quantum state $\ket{\psi_{t'}}$.

\para{Related Work.}
Radio transmitter localization using a set of sensors/receivers has been
widely studied~\cite{localization-guide, localization-survey, multi-tx-dyspan-19}. 
Localization methods can be roughly classified into two types: geometry-based and fingerprinting-based.
The geometry-based method includes multilateration (by measuring time-of-flight between the transmitter and multiple sensors) or triangulation (by measuring angle-of-arrival (AoA) of the transmitter at multiple sensors)~\cite{nsdi13-arraytrack}.
The fingerprinting-based method~\cite{infocom00-radar} has a training stage that records
the signal fingerprint for certain locations; Localization is then achieved by
matching the real-time signal to the recorded fingerprints. 
Here, a fingerprint for a transmitter location may be a vector of received signal strengths (RSS)~\cite{localization-guide} at the sensors.  
% In the training phase, a person holds a receiver and travels in an area, meanwhile RADAR records information about the radio signal as a function of the location, i.e., constructs a received signal strength (RSS) fingerprinting map.
% In the localization phase, RADAR conducts the nearest neighbor search and reports the location of the closest signal on the fingerprinting map.
%The premise is that RSS information provides a means of inferring RF location.
%%%%%%%%%%%%%%%%%
%The fingerprinting approach will work for arbitrary signal propagation models and other measurements, such as AOA.
% For simpler propagation models and/or 
% other measurements (e.g., angle/time of arrival, other
% techniques such as triangulation~\cite{nsdi13-arraytrack}, triliteration~\cite{} can be used.
%%%%%%%%%%%%%%%
Localization of simultaneously-active multiple transmitters is more challenging,
and has been addressed in recent works~\cite{wowmom21, ipsn20-mtl, pmc22-deepmtlpro}.
%%%%%%%%%%%%%%%%%%%%%%%

Recently, there have been some works that have used quantum technology to investigate
intruder/transmitter localization-related problems. E.g.,~\cite{lcn22-qloc} develops a scheme
to improve the {\em size} of the fingerprints used in the above-described 
fingerprinting approach, by encoding classical sensor data into qubits 
through quantum amplitude encoding.
%%%%%%%%%%%
In addition,~\cite{PR22-quantum_positioning} derives analytical equations to compute the AoA of an
incoming RF signal using four entangled distributed quantum sensors, without any evaluations.~\cite{qsn-detection} proposes a quantum sensor network 
using Mach-Zehnder interferometers to detect (not localize) 
intruders for surveillance purposes.
%%% our PRA is in arXiv, but not in the QCE submission
Finally,~\cite{Hillery_2023,qsn-acm-24} investigate the optimization of the initial state in discrete-outcome quantum sensor networks and show that an entangled initial state yields
higher measurement accuracy in some applications.
In particular, QuCS lecture series~\cite{zhiding2023} have successfully enhanced the visibility of quantum computing software and system level worldwide.
% The difference between ours and theirs is that we aim to use {\em both} quantum sensors and quantum localization algorithms, while the first work uses classical sensors and quantum localization algorithms, and the second work uses quantum sensors and classical localization algorithms.}

\softpara{Parameter Estimation using Quantum Sensors.}
Prior works on parameter estimation using quantum sensors
include: estimation of single~\cite{Giovannetti_2011} %~\cite{muri-20} 
or multiple independent parameters~\cite{mpe_2018}, estimation of a single linear function over parameters~\cite{Altenburg_2019}, and estimation of multiple linear functions~\cite{Rubio_2020}. Our transmitter localization problem can be looked upon
as a novel single parameter (TX location) estimation problem based on sensor measurements that are functions (based on signal propagation model and distance) of the parameter being estimated.
