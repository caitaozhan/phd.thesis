A quantum sensor (QS) is able to measure various physical phenomena with extreme sensitivity.
QSs have been used in several applications such as atomic interferometers, but few applications of a quantum sensor network (QSN) have been proposed or developed.
We look at a natural application of QSN---localization of an event (in particular, of a wireless signal transmitter).
In this paper, we develop effective quantum-based techniques for the localization of a transmitter using a QSN.

Our approaches pose the localization problem as a well-studied quantum state discrimination (QSD) problem and address the challenges in its application to the localization problem. 
In particular, a quantum state discrimination solution can suffer from a high probability of 
error, especially when the number of states (i.e., the number of potential transmitter locations in our case) can be high. 
We address this challenge by developing a two-level localization approach, which localizes the transmitter at a coarser granularity in the first level, and then, in a finer granularity in the second level. 
%%%%%%%%%%%%%%
We address the additional challenge of the impracticality of general measurements by 
developing new schemes that replace the QSD's measurement operator with a trained parameterized hybrid quantum-classical circuit.
%%%%%%%%%%%%%%%
Our evaluation results using a custom-built simulator show that our best scheme is able to 
achieve meter-level (1-5m) localization accuracy; 
in the case of discrete locations, 
it achieves near-perfect (99-100\%) classification accuracy. 