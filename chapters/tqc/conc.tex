\section{Conclusion and Future Directions}
\label{sec:tqc-conclusion}

In this work, we formulate the problem of initial state optimization in detector quantum sensor networks, which has potential applications in event localization.
We first derive the necessary and sufficient conditions for the existence of an initial state that can detect the firing sensor with perfect accuracy, i.e., with zero probability of error.
Using the insights from this result, we derive a conjectured optimal solution for the problem and provide a pathway to proving the conjecture.
Multiple search-based heuristics are also developed for the problem and the heuristics' numerical results successfully validate the conjecture.
In the end, we extend our results to the unambiguous discrimination measurement scheme, non-uniform prior, and considering quantum noise.

Beyond proving the stated Conjectures in the paper, there are many generalizations of the addressed
\iso problem of great interest in terms of: 
(i) More general final states (e.g, two sensors may change at a time, allowing for multiple impact operators $U_1, U_2$, etc.), 
(ii) Restricting the measurement operators allowed (e.g., allowing only the projective measurements and/or local measurements~\cite{umd-entanglement}), to incorporate practical considerations in the implementation of measurement operators. 
We are also interested in proving related results of interest, e.g., \iso initial-state solution being the same for minimum error and unambiguous discrimination.

