\section{Related Work}
\label{sec:wowmom-related}
\para{Spectrum sensing} is usually being realized by some distributed crowdsourced low-cost sensors. 
Electrosense~\cite{electrosense} and its follow up work Skysense \cite{mobisys20-skysense} are typical work of spectrum sensing.
In this crowdsourced sensing paradigm~\cite{chakraborty2017specsense}, sensors collect I/Q samples (in-phase and quadrature components of raw signals) and compute PSD (power spectral density), which is RSS.
Crowdsourced low-cost sensors do not have the capability to collect AoA (angle of arrival) data because it requires more expensive antenna arrays.
They also do not have the capability to collect ToA (time of arrival) data because it requires the transmission of a wide-band known sequence~\cite{pimrc2021-localize}, which is impossible in the case of localizing (blind) intruders.
Spectrum sensing platforms serve as the foundation of the spectrum applications, and transmitter localization is one of the main applications.
Other applications include signal classification~\cite{toccn18-sigclassify}, spectrum anomaly detection~\cite{ben-zhao}, sensor selection~\cite{ton-sensorselect,bhattacharya2022fast}, spectral occupancy estimation~\cite{mobicom21-deepradar}, etc.

\para{Transmitter localization.} Localization of an intruder in a field using sensor observations has been widely studied, but most of the works have focused on
localization of a single intruder~\cite{arani2018,dutta2016see}.
%%%%%
In general, to localize multiple intruders, the main challenge comes
from the need to ``separate'' powers at the sensors~\cite{mobicom-30},
i.e., to divide the total received power into power received from
individual intruders. Blind source separation is a very challenging
problem; only very limited settings allow for known
techniques using sophisticated receivers~\cite{freq-sig,ben-zhao}.
%%%
We note that (indoor) localization of a
  device~\cite{infocom00-radar} based on signals received from multiple reference points (e.g, WiFi access
  points) is a quite different problem
  (see~\cite{zafari-19} for a recent survey), as the signals from
  reference points remain separate, and localization or tracking of multiple
  devices can be done independently.
  Recent works on multi-target localization/tracking such as~\cite{ipsn19-multipassive} are different in the way that targets are passive, instead of active transmitters in the \mtl problem.
Among other related works,~\cite{multi-tx-dyspan-19} addresses the challenge of handling time-skewed sensors observations in the MTL problem.
%%%%%%%%%%%%%%%%%%%%%%%
\eat{In absence of blind separation methods, to the best of our knowledge,
only a few works have addressed multiple intruder(s) localization. 
In particular,
(i)\splot~\cite{mobicom17-splot} decomposes the multi-transmitter
localization problem to multiple single-transmitter localization
problems based on the sensors with highest of readings in a
neighborhood, (ii)~\cite{clustering} uses a clustering-based approach, (iii)~\cite{Quasi-EM} uses an
EM-based approach, (iv) \map~\cite{ipsn20-mtl}, uses 
a hypotheses-based Bayesian
approach in conjunction with a divide-and-conquer strategy to first localize 
``isolated"  intruders and then localize the remaining intruders,
(v)We note that the techniques of~\cite{mobicom17-splot,Quasi-EM} assume a propagation
model, while that of~\cite{clustering,Quasi-EM} require a priori
knowledge of the number of intruders present. 
Schemes in~\cite{Quasi-EM} and~\cite{clustering} have been shown inferior in performance in~\cite{mobicom17-splot,ipsn20-mtl}.
}

\para{Wireless localization} techniques mainly fall into two categories: geometry mapping and fingerprinting-based.
Geometry mapping mainly has two subcategories: ranging-based such as trilateration (via RSS/RSSI, ToA~\cite{zongxing2022tracking}, TDoA) and direction-based such as triangulation (via AoA).
Fingerprinting-based methods can use all signal physical measurements including but not limited to amplitude, RSS/RSSI, ToA, TDoA, and AoA.
Whenever deep learning is used for localization, it can be considered as a fingerprinting-based method, since it requires a 
training phase to survey the area of interest and a testing phase to search for (predict) the most likely location.

\para{Deep learning for wireless localization}. 
Quite a few recent works have harnessed the power of deep learning in the general topic of localization.
E.g., DeepFi in~\cite{DeepFi2016} designs a restricted Bolzmann machine that localizes a single target using WiFi CSI amplitude data. 
DLoc in~\cite{mobicom20-deeploc} uses WiFi CSI data as well. 
Its novelty is to transform CSI data into an image and then uses an image-to-image translation method to localize a single target.~MonoDCell 
in~\cite{sigspatial19-monodcell} designs an LSTM that localizes a single target in indoor environment using cellular RSS data.
~\cite{pimrc2021-localize} designs a three-layer neural network that locations a single transmitter.~\deeptx in~\cite{icccn20-deeptxfinder} 
uses CNN to address the same \mtl problem using RSS data in this chapter.

\para{Transmitter Power Estimation.} There are several works that estimate the transmission power of a single transmitter.
~\cite{PowerEstimate2010Zafer} studies the ``blind" estimation of transmission power based on received-power measurements at 
multiple cooperative sensor nodes using maximum likelihood estimation. Blind means there is no prior knowledge of the location of the 
transmitter or transmit power.~\cite{Ureten2011powerlocation} propose an iterative technique that jointly estimate the location and 
power of a single primary transmitter.
In~\cite{icoin2007-powerposition}, the primary transmitter location and power is jointly estimated by a constrained optimization method.
~\cite{ipsn20-mtl} uses the maximum likelihood estimation method to estimate the power of an isolated single transmitter and 
adopts an online learning method to estimate the power of multiple close by transmitters simultaneously.

\para{Machine learning empowered applications.}
Machine learning can be broadly classified into supervised, unsupervised, semi-supervised and reinforcement learning~\cite{yifei2022,yifei2022path}.
~\cite{yifei2023} has studied reinforcement learning in the setting of episodic Markov decision processes.
~\cite{yifei2021} studies the off-policy evaluation problem in reinforcement learning with linear function approximation.
Machine learning techniques have empowered many applications.
XGBoost is great at capturing the upward trend using the portfolio features constructed by PolyModel theory~\cite{siqiao2023}.
Novel meta-learning algorithms are developed for generalizable magnetic resonance imaging reconstruction~\cite{wanyu_thesis}.
Deep reinforcement learning can be novely used to minimize battery energy storage cost~\cite{binhuang2023grid} 
and maintain the fidelity of equivalent model for active distribution networks~\cite{binhuang2023renew}.
~\cite{xiaobo_thesis} proposes machine learning methods that help transportation engineers and policymakers conduct accurate traffic performance ealuations~\cite{xiaobo2020,xiaobo2021,xiaobo2023}.
Novel machine learning techniques have proposed to deal with noisy labeled data~\cite{jinjin2023} and genetic data~\cite{jinjin_thesis}.
Social media bots can be detected effectively via behavioral patterns~\cite{wu2023botshape} and metric learning~\cite{wu2023bottrinet}.
AI-assisted audio command recognition enables collaborative human-robot drone inspection of bridges~\cite{yuli_thesis}.
In~\cite{ziheng_thesis,ziheng2022}, algorithms are developed to make machine learning models more transparent, accountable and explainable~\cite{ziheng_relax,ziheng2023dark}.
Network function virtualization~\cite{wang2023thesis,wang2022quadrant,wang2020slos} has great potential in low-performance edge devices~\cite{wang2023scheduling}
as more applications and ISP functionalities are expected to be offloaded to edge clouds~\cite{wang2021galleon,wang2023pinolo}.
Machine learning can help the simulation and analyses of multi-platelet recruitment simulation~\cite{yicong_thesis}.
In~\cite{peineng2021,peineng2021semi}, semi-supervised learning algorithms are developed to segment platelet images.
In~\cite{peineng2023}, a machine learning-guided imaging approach is used to segment platelet geometries and quantify adhesion dynamics parameters.
DeepVS~\cite{zongxing2022dl} combines 1D CNN and attention models to exploit local features and temporal correlations to improve RF-based vital signs sensing.
A LSTM encoder-decoder model is proposed to generate Chinese poetry~\cite{yubo2017text}.
Machine learning also has applications in e-commerce~\cite{kexin2023ecommerce}, recommendation system~\cite{dong2020,dong2023,dong2023recommend}
financial time series data~\cite{kexin2023financial}, and spectrum allocation~\cite{mohammad2024}.

\para{Mobile health sensing and edge computing.}
RF sensing enables some important mobile health applications such as heart and respiratory rate monitoring~\cite{zongxing2022uwb,zongxing2022measure}. 
RF based solutions support practical and longitudinal respiration monitoring owing to their non-invasive nature~\cite{zongxing2023rf,zongxing2021uwb}.
~\cite{zongxing2024rfq} proposed a robust RF-based respiration monitoring.
Low quality RF sensing data will negatively affect the sensing task, thus reliable signal quality detection is crucial~\cite{zongxing2021quality}.
Apart from RF sensing, acoustic sensing can also enable important applications such as face authentication~\cite{zongxing2019face,zongxing2022face}.
Apart from RF sensor, other sensor such as inertial sensor, photoplethysmography and actigraphy also play a big role in mobile health and wearable IoT, 
such as finger motion tracking~\cite{yilin2021}, end-stage kidney fluid intake prediction in~\cite{guimin2022health} and predicting salivary cortisol levels in pancreatic cancer patients~\cite{guimin2021pancreatic}.
In~\cite{guimin2021health}, a semi-supervised graph instance transformer is proposed for mental health inference.
Graph neural networks are useful in various IoT sensing environment~\cite{guimin2023survey} and mobile health sensing~\cite{guimin_thesis}.
In traditional wireless sensor networks, the sensing data is uploaded (via wireless) to a centralized server~\cite{gupta2020,yubo2023blockchain} and the server process the sensing data.
In an emerging computing paradigm called edge computing, however, sensing data is mainly processed locally on the resource constrained sensors, 
e.g., on-device machine learning~\cite{yubo2020ondevice,yubo2022demo,yubo2019ondevice}.
Thus, various works target to scale up task execution on resource-constrained systems~\cite{yubo_thesis}, such as 
SmartOn~\cite{yubo2021smarton}, Antler~\cite{yubo2023efficient} and intermittently-powered systems~\cite{yubo2023audio,yubo2023intermittent}.
Works tackle the computation overhead and battery limitation of on-device edge computing via 
interactive learning framework~\cite{zhou2022}, offloading scheme~\cite{zhou2023offloading} and reinforcement-learning-based scheduling technique~\cite{zhou2023}.
Studies have shown that deep learning can reconstruct ambiance information~\cite{wenwan} for mental health purposes.
Large language models could help building a practical benchmark for cloud configuration generation~\cite{yuning2023cloud}.
Other works study the non-volatile hybrid memory in mobile memory systems~\cite{feiwen2021,feiwen2021fpga,feiwen2022}.

%Both DLoc and \deeptx also represent the explore the usage of CNN in the wireless localization problem by representing the input wireless data as images.
%We also use images to represent wireless data (see Fig. \ref{fig:input}) and harness the power of deep learning on top of that.
%Below, we introduce our high-level approach.
  
  %, and (ii)~\cite{info-20} that addresses the sensor selection optimization problem.
  

  
%%%%%
%\blue{Other related works include:~\cite{mobicom-22} where sensors are
%  on mobile and controlled robots,~\cite{mobi-25} focusses on spectrum
%  allocation via spectrum hole detection in presence of background 
%  transmitters.} \red{HG: Remove this sentence?}

%% Online selection of sensors: ipsn-04, .... latency vs. energy .. since,
%% latency is equally critical, ... we dont want to run it for every intruder ... 
%% Similarly,
%% \cite{krause2008near} shows that minimizing uncertainty in a gaussian
%% process is submodular, and thus greedy selection provides a bounded
%% solution to the optimal.

%% Multiple studies have studied sensor selection to maximize the
%% accuracy of detection of some event \cite{rowaihy2007survey}.

%% For
%% example, \cite{joshi2009sensor} provides a heuristic for sensor
%% selection by forming a convex optimization problem.  However, it uses
%% a different metric to measure the accuracy of detection.
%% %%%%
%% Other studies, such as
%% \cite{shamaiah2010greedy} and \cite{bian2006utility} have proposed
%% leveraging submodularity to select sensors.  

%% There are also studies in the active learning literature that focus on
%% online selection.  For example, \cite{yuxin-when} limits the mutual
%% information while selecting the minimum possible number of sensors.
%% \cite{krause2012near} shows that mutual information in sensor
%% selection is submodular in the absence of noise and propose a
%% probabilistic greedy algorithm by leveraging it.
%% \cite{golovin2011adaptive} proposes the concept of adaptive
%% submodularity that generalizes the greedy approximation to online
%% selection. Our online selection algorithm builds upon these studies to
%% limit the number of sensors while maximizing the mutual information.
%% However, in the presence of noise, mutual information is not adaptive
%% submodular in nature.  Thus, our work modifies the algorithm discussed
%% in \cite{yuxin-when} to make it suitable for our use case.
