\subsection{Related Work}
\label{sec:ipsn-related}

Localization of an intruder in a field using sensor observations has
been widely studied, but most of the works have focused on
localization of a single intruder~\cite{arani2018,dutta2016see}.
%%%%%
In general, to localize multiple intruders, the main challenge comes
from the need to ``separate'' powers at the sensors~\cite{mobicom-30},
i.e., to divide the total received power into power received from
individual intruders. Blind source separation is a very challenging
problem; only very limited settings allow for known
techniques~\cite{freq-sig,ben-zhao} using sophisticated receivers. In
our context of hypotheses-driven approach, the challenge of source
separation manifests in terms of a large number of hypotheses, a
challenge addressed in~\S\ref{sec:map-time}.
%%%
We note that (indoor) localization of a
  device~\cite{infocom00-radar} based on signals received from multiple reference points (e.g, WiFi access
  points) is a quite different problem
  (see~\cite{zafari-19} for a recent survey), as the signals from
  reference points remain separate, and localization or tracking of multiple
  devices can be done independently.
  Recent works on multi-target localization/tracking are different in the way that targets are passive~\cite{ipsn19-multipassive, ipsn19-chorus, ipsn19-snaploc}, instead of active transmitters in this work.

In absence of blind separation methods, to the best of our knowledge,
only a few works have addressed multiple intruder(s) localization, and
none of these consider it in the presence of a dynamically changing
set of authorized transmitters. In particular,
(i)~\cite{mobicom17-splot} decomposes the multi-transmitter
localization problem to multiple single-transmitter localization
problems based on the sensors with highest of readings in a
neighbohood, (ii)~\cite{clustering} works by clustering the sensors
with readings above a certain threshold and then localizing intruders
at the centers of these clusters, (iii)~\cite{Quasi-EM} uses an
EM-based approach.
%%%
The techniques of~\cite{mobicom17-splot,Quasi-EM} assume a propagation
model, while that of~\cite{clustering,Quasi-EM} require a priori
knowledge of the number of intruders present.  We have compared our
approach with~\cite{mobicom17-splot,clustering} in \S\ref{sec:eval},
while~\cite{Quasi-EM} has high computational cost and has also been
shown to be inferior in performance
to~\cite{mobicom17-splot,clustering} even for a small number of
intruders. Other related works include
  (i)~\cite{multi-tx-dyspan-19} that addresses the challenge of
  handling time-skewed sensors observations in the MTL problem, and
  (ii)~\cite{info-20} that addresses the sensor selection optimization
  problem for our proposed hypotheses-based localization approach.
  
%%%%%
%\blue{Other related works include:~\cite{mobicom-22} where sensors are
%  on mobile and controlled robots,~\cite{mobi-25} focusses on spectrum
%  allocation via spectrum hole detection in presence of background 
%  transmitters.} \red{HG: Remove this sentence?}

%% Online selection of sensors: ipsn-04, .... latency vs. energy .. since,
%% latency is equally critical, ... we dont want to run it for every intruder ... 
%% Similarly,
%% \cite{krause2008near} shows that minimizing uncertainty in a gaussian
%% process is submodular, and thus greedy selection provides a bounded
%% solution to the optimal.

%% Multiple studies have studied sensor selection to maximize the
%% accuracy of detection of some event \cite{rowaihy2007survey}.

%% For
%% example, \cite{joshi2009sensor} provides a heuristic for sensor
%% selection by forming a convex optimization problem.  However, it uses
%% a different metric to measure the accuracy of detection.
%% %%%%
%% Other studies, such as
%% \cite{shamaiah2010greedy} and \cite{bian2006utility} have proposed
%% leveraging submodularity to select sensors.  

%% There are also studies in the active learning literature that focus on
%% online selection.  For example, \cite{yuxin-when} limits the mutual
%% information while selecting the minimum possible number of sensors.
%% \cite{krause2012near} shows that mutual information in sensor
%% selection is submodular in the absence of noise and propose a
%% probabilistic greedy algorithm by leveraging it.
%% \cite{golovin2011adaptive} proposes the concept of adaptive
%% submodularity that generalizes the greedy approximation to online
%% selection. Our online selection algorithm builds upon these studies to
%% limit the number of sensors while maximizing the mutual information.
%% However, in the presence of noise, mutual information is not adaptive
%% submodular in nature.  Thus, our work modifies the algorithm discussed
%% in \cite{yuxin-when} to make it suitable for our use case.
