\newcommand{\id}[1]{\ensuremath{\mathit{#1}}}
% \newcommand{\tr}{\id{tr}}
\newcommand{\ttohp}{\pi}

\section{LP Formulation for the \qnr Problem}
\label{sec:swapping_multiple-path}

In this section we provide an  optimal \LP-based solution to
the \qnr problem.  Although polynomial-time, this solution has high complexity, so its main use is as a benchmark in evaluating the more efficient (but possibly sub-optimal) algorithms for the problem.  

Our approach follows from the observation each
swapping tree in a QN can be viewed as a special kind of path (called
\emph{B-hyperpath}~\cite{Beckenbach2019}) over a hypergraph
constructed from the network graph.
%Consequently, optimal routing using multiple swapping trees is naturally formulated as
%network flow over directed hypergraphs.
%, analogous to how optimal routing over
%multiple linear paths in the
%classical case is formulated in terms of ordinary directed graphs.
%Optimal network flow over directed hypergraphs
%is consequently solved via \LP.  
%
We begin by describing the hypergraph construction for the
single-pair case and ignoring fidelity constraints. We then extend
traditional hypergraph-flow algorithm to incorporate losses
(e.g., due to BSM failures), stochasticity, and the interaction between 
memory constraints and stochasticity. Finally, we extend the formulation 
to multiple $(s,d)$ pairs and incorporate fidelity constraints.  
%This approach yields a polynomial-time solution, albeit
%with a large exponent.  We also show that the optimal solution cannot
%be obtained by greedy iterative \DP.

Optimal generation of long-distance entanglement was posed as an \LP
problem in~\cite{Daietal2020}, but differs from our \blue{more general formulation work} 
in three main
ways. First of all,~\cite{Daietal2020} assumes unbounded memory capacity at each
swapping node to queue up incoming \epss. In contrast, our model
has bounded memory capacity at each node, and consequently, 
our \LP formulation deals with \textit{expectations} over
rates/latencies rather than scalar rate values. 
Secondly, our formulation accounts for node capacity constraints in addition to link constraints. 
Thirdly, our formulation poses the problem in terms of hypergraph flows, which permits us to easily incorporate fidelity and decoherence constraints. 
%%%%%%%%%%%%%%%%%%%%%%%%%

\subsection{Hypergraph-Based Representation of Entanglement Generation}
We begin by recalling standard hypergraph notions~\cite{Beckenbach2019,GalloEtAl1993,ThakurTripathi2009}.

% \begin{definition}{Hypergraph}
\label{defn:hypergraph}
\rm
  A directed hypergraph $H = (V(H),$ $E(H))$  has a set of vertices $V(H)$
  and a set of (directed) \emph{hyperarcs} $E(H)$, where each hyperarc $e$ is a
  pair $(t(e), h(e))$ 
  of non-empty disjoint subsets of $V(H)$.  
A \emph{weighted}
  hypergraph is additionally equipped with a weight function $\omega: E(H)
  \rightarrow R^+$.  
% \end{definition}

Sets $t(e)$ and $h(e)$ are 
  called the \emph{tail} and \emph{head}, resp., of hyperarc $(t(e),
  h(e))$.
A hyperarc $e$ is a \emph{trivial} edge if both $t(e)$ and $h(e)$ are
  singleton; and \emph{non-trivial} otherwise. 
A hyperarc $e$ where $|h(e)| = 1$, i.e. whose head is singleton, is called a $B$-arc.   A hypergraph consisting only of $B$-arcs is called a $B$-hypergraph.

 
% \begin{definition}{Connectivity and $B$-Hyperpaths}
\label{defn:hyperpath}
\rm
A vertex $t$ is \emph{$B$-connected to} vertex $s$ in hypergraph $H$ if
$s=t$ or there is a hyperarc $e \in E(H)$ such that $h(e) = \{t\}$ and
every $v \in t(e)$ is $B$-connected to $s$ in $H$.
%
A $B$-hyperpath from $s$ to $t$ is a minimal $B$-hypergraph $P$ 
such that $V(P) \subseteq V(H)$,  $E(P) \subseteq E(H)$, and $t$ is
$B$-connected to $s$ in $P$.
% \end{definition}

\begin{figure}
    \centering
    \includegraphics[width=0.8\textwidth]{figures/swappingtrees/hypergraph.jpg}
    \caption{ST-hypergraph for a 4-node linear network. Not all $prod$ nodes are shown.}
    \label{fig:swapping_LP}
\end{figure}

\para{ST-hypergraph.}
Given a QN and single $(s,d)$ pair, we first construct a hypergraph 
that represents the set of all possible swapping trees rooted
at $(s,d)$.
Given a QN represented as an undirected graph $G=(V,E)$
and a single $(s,d)$ pair, its \emph{ST-hypergraph} 
is a hypergraph $H$ constructed as follows  (see 
Fig. \ref{fig:swapping_LP}). All pairs of 
vertices below are unordered pairs.
\begin{itemize}
\item $V(H)$ consists of:
  \begin{enumerate}
  \item Two distinguished vertices $\id{start}$ and $\id{term}$
  \item $\id{prod}(u,v)$ and $\id{avail}(u,v)$ for all distinct $u, v \in
    V$
  \end{enumerate}
\item $\id{E(H)}$ consists of 5 types of hyperarcs:
  \begin{enumerate}
  \item\ [Start] $e = (\{\id{start}\}, \{\id{avail}(u,v)\}) \ \ \forall u,v \in V$.
  \item\ [Swap] $e = (\{\id{avail}(u,w),  \id{avail}(w,v)\}, \{\id{prod}(u,v)\})$, for all \emph{distinct} $u,v,w \in V$.
  \item\ [Prod] $e = (\{\id{prod}(u,v)\}, \{\id{avail}(u,v)\})\ \ \forall u,v \in V$.
  \item\ [Term]  $e = (\{\id{avail}(s,d)\}, \{\id{term}\})$.
  \end{enumerate}
 \end{itemize}
In an ST-hypergraph, vertices $\id{start}$ and $\id{term}$
represent source and sink nodes of a desired hypergragh-flow (see below).
Other vertices represent \epss over a pair of nodes in $G$.
%%%%%%%%%%%%%%%%%%%%%%%%
Hyperarcs
represent how the tail \epss contribute to that at the head. 
For ease of accounting, we categorize generated \epss using different
types of vertices:
%%%%%%%%%%%%%%%%%%%%%
$\id{start}$ represents link-level \epss generated over links in $G$, 
%
$\id{prod}$ represent \epss produced by atomic entanglement-swapping, 
%
and $\id{avail}$ represent \epss generated from either of the above.
%%%%%%%%%%%%%%%%%%%%%%%%%%%%%%%%%
``Start'' and ``Prod'' arcs turn the $\id{start}$ and 
$\id{prod}$ \epss respectively into  $\id{avail}$ \epss and 
thus make them available for further swapping. 
%%%%%%%%%%%%%%%%%%%%
``Swap'' arcs represent swapping over the triplets of nodes $(u,w,v)$.
%%%%%%%%%%%%%%%%%%%%%
Note that an ST-hypergraph is a $B$-hypergraph, as ``Swaps''
are the only non-trivial hyperarcs, and their head is singleton.

\softpara{Swapping Trees as $B$-Hyperpaths.} 
Given a \qnr problem with a single pair $(s,d)$, it is easy to see
that any swapping tree generating $(s,t)$ \epss can be represented 
by a unique $B$-hyperpath from \emph{start} to \emph{term} in the above 
ST-hypergraph. Thus, it easily follows that 
a \qnr problem of selection of (multiple)
swapping trees is equivalent to finding an optimal hypergraph flow 
from $\id{start}$ to $\id{term}$ in $H$. Note that $H$ has $O(|V|^2)$
vertices and $O(|V^3|)$ hyperarcs.

\subsection{Entanglement Flow as \LP}

%Hypergraph arc-flow problem.  \LP formulation.  Incorporate $\nu_0$ and $2/3$ in this setting.
We now develop an LP formulation to represent the \qnr problem over $(s,d)$ in $G$
as a hypergraph-flow problem in $H$.
%%%%%%%%%%%
In contrast to the classic hypergraph-flow formulation~\cite{Beckenbach2019}, 
we need to consider \emph{lossy} flow, with loss arising
from two sources: (i) \es operations have a given success probability, and (ii) waiting for both qubits to arrive before performing \es leads to losses since the arrival of \epss follow independent probability distributions.
% With limited memory, the distribution of arrivals mean that some arriving EPs will have to be discarded, hence the loss.
For the latter, we make use of Observation~\ref{ob:swapping_expdist}.
%%%%%%%%%%%%%%%%%%%%%%%%%%
The proposed LP formulation is as follows.
\begin{itemize}
\item \textbf{Variables}: $z_a$,  for each
  hyperarc $a$ in $H$, represents the \eps generation rate
  over each of the (one or two) node-pairs in $a$'s tail.
  This enforces the condition that \eps rates over the two 
  node-pairs in $\id{prod}$ hyperarc's tail are equal. Thus, the \LP 
  solution will result in \emph{throttled} swapping trees.
  
\item \textbf{Capacity Constraints}: $z_a \in R^+$ for all hyperarcs $a$ in $H$. 
We use 
Eqn.~\eqref{eqn:swapping_qnr-1}
% Eqn~\ref{eqn:qnr-1}
to add the following constraints due to nodes in $G$. 
\begin{eqnarray*}
%1/\et &\geq& z_a/(\red{\gp}\ep^2\php)    \quad \forall a = (\id{start}, \id{avail}(u,v)) \in E(H).  \\
1/\gt &\geq& \sum_{x \in E(i)} z_{a(x)}/(\gp^2\ep^2\php) \quad  \forall i \in V.
\end{eqnarray*}
Above $a(x)$ is the hyperarc in $H$ of the form $(\id{start}, \id{avail}(x))$ 
where $x$ is an edge in $G$. 

\item \textbf{Flow Constraints} which vary with vertex types. Below, we use
notations $\id{out}(v)$ and $\id{in}(v)$ to represent outgoing and incoming
hyperarcs from $v$. Formally, $\id{out}(v)$ is $\{a \in E(H): v
\in t(a)\}$ and $\id{in}(v)$ is $\{a \in E(H): v \in h(a)\}$.

  \begin{itemize}
  \item For each vertex $v$ s.t. $v=\id{avail}(\cdot)$:
\[
  \sum_{a \in \id{in}(v)}  z_a = \sum_{a' \in \id{out}(v)} z_{a'}
\]
That is, there is no loss in making already generated entanglements
available for further swapping.

  \item For each vertex $v$ s.t. $v=\id{prod}(\cdot)$:
\[
  \sum_{a \in \id{in}(v)}  z_a\bp(2/3) = \sum_{a' \in \id{out}(v)} z_{a'}
\]
The $(2/3)\bp$ factor follows from Observation~\ref{ob:swapping_expdist}, and
accounts for loss due to swapping failures as well as due to waiting for
arrival of both \epss for swapping.
  \end{itemize}
\item \textbf{Objective}: Maximize $\sum_{a \in \id{in}(\id{term})} z_a$
%\[
%\sum_{a \in \id{in}(\id{term})} z_a
%\]
%Vertex $\id{term}$ represents an external consumer of \epss, and hence
%the \LP formulation maximizes the rate of \epss over $(s,d)$ in $G$.
\end{itemize}

\para{Multiple-Pairs Multi-Path:}
The above \LP formulation for the single-pair \qnr problem can be 
readily extended to the multiple-pairs case.
Let $\{(s_1, d_1), (s_2, d_2), \ldots, (s_n,d_n)\}$ be a set of
source-destination pairs. The \emph{only change} is
that the hypergraph $H$ now has $n$ arcs $(\{\id{avail}(s_i,d_i)\},
\{\id{term}\})$ for all $i$.  The other arcs model the generation of EPRs 
independent of the pairs, and thus are unchanged. 
It is interesting to note that the multi-pairs problem, 
typically formulated as multi-commodity flow in classical networks, 
is posed here as single-commodity flow over hypergraphs.

\subsection{Fidelity}

Constraints on loss of fidelity due to noisy BSM operations
and from decoherence due to the age of qubits can be added to the
\LP formulation, as follows.  
%%%%%%%%
Recall that constraint on operation-based fidelity loss is modelled
by limiting the number leaves of the swapping tree, and in~\S\ref{sec:swapping_dec}, 
we formulated the decoherence constraint by limiting the
heights of the left-most and right-most descendants  of the root's
children. These
\emph{structural} constraints on swapping trees can be lifted to the
\LP formulation by adding the leaf count and heights as
parameters to $\id{prod}$ and $\id{avail}$ vertices; and (ii) swapping
the \epss generated from only the compatible vertices.  

In particular, we generalize the ST-hypergraph to a
\emph{fidelity-constrained} one called $H^{(F)}$, where the
$\id{prod}$ and $\id{avail}$ vertices are parameterized by $u,v \in
V$, and in addition by $(n,h)$ where $n$ is the number of leaves 
and $h=(h_{ll}, h_{lr}, h_{rl}, h_{rr})$ represents the depths of left-most 
and right-most descendants of the root's children, of the
trees rooted at $(u,v)$ with those parameter values.
%%%%%%%%%%%%%%%%%%%%
%%%%%%%%%%%%%%%
In terms of edges, the most interesting difference $H^{(F)}$
and $H$ is in ``Swap'' edges.   In 
$H^{(F)}$, ``Swap'' edges are 
$( \{\id{avail}(u,w,n',h') $
$\id{avail}(w,v,n'',h'')\}, $
$\{\id{prod}(u,v,n,h)\})$ only if
$n = n'+n''$, and $h,h',h''$ are such that
$h_{ll} = h'_{ll}+1$, $h_{lr} = h'_{rr}+1$, $h_{rl}
= h''_{ll}+1$ and $h_{rr} = h''_{rr}+1$.  
The above constraints ensure that only compatible subtrees are
composed into bigger trees. The other changes are for bookkeeping: 
``Gen'' are from $\id{gen}(u,v)$ to $\id{avail}(u,v,1,(0,0,0,0))$; 
``Prod''  are from $\id{prod}(u,v,n,h)$ to
$\id{avail}(u,v,n,h)$; and finally ``Term''
are from $\id{avail}(s,t,n,h)$ to
$\id{term}$ for $n \leq \fidl$ and $h$ such that 
$f(h) \leq \fidd$; here, $f(h)$ gives the tree's age based on $h$ 
values (following \S\ref{sec:swapping_dec}) 
while using the link rates based on 50\% node-capacity 
usage.

%%%%%%%%%%%%%%%%%

\eat{
Note that the new
parameter $n$ is a natural number in $[0, N_F]$ where $N_F$ is the
number of leaves that will satisfy the given constraint on
operation-based fidelity loss (see~\cite{Briegel98}, also~\cite[Eq. 2,
p.nn]{Delft}). \red{What about the ranges of $h_{ll}$ etc?}



At a high level, fidelity is a measure of how close an entangled pair
of qubits is to the ideal of maximal entanglement.  Every entanglement
swap operation decreases fidelity.  Briegel et al~\cite{Briegel98}
derives the fidelity of the entanglement generated via swapping ($F'$)
to the fidelities of the elementary entanglements ($F$).   This
formulation considers the noise introduced by swapping operations and
the number of elementary entanglements used to generate the final
one\footnote{A similar formulation assuming the swapping
operations introduce no noise is in~\cite{Delft}.}.  
Of most importance to us is the fact that final fidelity is
independent of the order of swapping operations.  Assuming that the
link-level EPRs are all generated with the same fidelity (an
assumption that is common, see~\cite{Delft}), then the fidelity of a
long-distance EPRgenerated by a swapping tree depends only on the
number of leaves of the tree.  Given a specific fidelity threshold $F$
for each $(s,t)$ demand, we can find the maximum number $N_F$ of leaves in a
swapping tree that will satisfy the demand (see also~\cite[Eq. 2,
p.nn]{Delft}). \red{details?}
}

\eat{
The ST-hypergraph with fidelity constraints is constructed
along the same lines as that from
Defn.~\ref{defn:ent-hg}, adding an extra parameter to $\id{avail}$ and
$\id{prod}$ vertices to keep track of number of leaves of their
corresponding trees.  Specifically, fidelity-constrained entanglement
hypergraph $H^{(F)}=(Hv, He)$ such that 
\begin{itemize}
\item $\id{Hv}$ consists of 
  \begin{enumerate}
  \item $\id{gen}(u,v)$ for each $(u,v) \in E$, as well as two distinguished vertices $\id{start}$ and $\id{term}$
  \item $\id{prod}(u,v,n)$ and  $\id{avail}(u,v,n)$ for each pair of distinct vertices $u, v \in
    V$, and $n \in [1,N_F]$.
  \end{enumerate}
\item $\id{He}$ consists of 5 types of hyperarcs:
  \begin{enumerate}
  \item\ [Start] $e = (\{\id{start}\}, \{\id{gen}(u,v)\})$ for each edge
    $(u,v) \in E$, with $\omega_h(e) = \omega(e)$.
  \item\ [Swap]  $e = (\{\id{avail}(u,w,n_1),  \id{avail}(w,v,n_2)\}, \{\id{prod}(u,v,n)\})$, for all
    \emph{distinct} $u,v,w \in V$ and $n_1, n_2, n \in [1,N_F]$ such
    that $n_1 + n_2 = n$,
with $\omega_h(e) = \inf$.
  \item\ [Gen] $e = (\{\id{gen}(u,v)\}, \{\id{avail}(u,v,1)\})$, for all
    $(u,v) \in E$ with $\omega(e)= \inf$.
  \item\ [Prod] $e = (\{\id{prod}(u,v,n)\}, \{\id{avail}(u,v,n)\})$ for all
    distinct $u,v\in V$ and $n \in [1,N_F]$, with $\omega(e)= \inf$.
  \item\ [Term] $e = (\{\id{avail}(s,t,n)\}, \{\id{term}\})$ for all $n \in
    [1,N_F]$ with $\omega(e)= \inf$.
  \end{enumerate}
\end{itemize}
The leaf-counting constraint in the $\id{avai}-\id{prod}$ hyperarcs
ensures that only trees with fewer than $N_F$ leaves are represented
by $H^{(F)}$.  
}

\eat{
Notably, there is no further change needed to incorporate fidelity
constraints; max $(s,t)$-arc flow of $H^{(F)}$ is defined by the same \LP
formulation as before.

\red{Note about purification?}

\subsection{Decoherence}
\red{TBD}
}


% [Gallo et al., 1993] Gallo, G., Longo, G., Pallottino, S., and
% Nguyen, S. (1993). Directed Hypergraphs and Applications. Discrete
% Applied Mathematics, 42(2):177–201.


% [Thakur and Tripathi, 2009] Thakur, M. and Tripathi,
% R. (2009). Linear Connectivity Problems in Directed
% Hypergraphs. Theoretical Computer Science, 410(27):2592–2618.
