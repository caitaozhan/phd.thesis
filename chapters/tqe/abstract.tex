Quantum network communication is 
challenging, as the No-cloning theorem in quantum regime
makes many classical techniques inapplicable; \tqbl{in particular, 
direct transmission of qubit states over long distances is infeasible
due to unrecoverable errors.}
%%%%%%%%%%%%%%%%%
For long-distance communication \tqbl{of unknown quantum states}, 
the only viable communication approach \tqbl{(assuming local operations
and classical communications)} is 
teleportation of quantum states, which requires a prior distribution of 
entangled pairs (\epss) of qubits.
Establishment of \epss across remote nodes can incur significant 
latency due to the low probability of success of the underlying 
physical processes. 
The focus of this chapter is to develop
efficient techniques that minimize \eps generation latency. Prior works
have focused on selecting entanglement \textit{paths}; in contrast, we\eat{propose
selection of efficient} \bleu{select}  \emph{entanglement swapping trees}---a more accurate 
representation of the entanglement generation structure. 
~\cite{swapping-tqe-22} has developed a dynamic programming algorithm 
to select an optimal swapping-tree for a single pair of nodes, under the given capacity
and fidelity constraints. For the general setting,~\cite{swapping-tqe-22} also developed an 
efficient iterative algorithm to compute a set of swapping trees.
However, the dynamic programming algorithm has a high time complexity, and thus, may not
be practical for real-time route finding in large networks.
In this chapter, we focus on developing an \emph{almost linear time} heuristic for the \spp problem,
based on the classic Dijkstra shorted path algorithm.
The designed heuristic performs close to the DP-based algorithms in our empirical studies.